% ----------------------------------------------------------
% Introdução (exemplo de capítulo sem numeração, mas presente no Sumário)
% ----------------------------------------------------------

\chapter{Introdução}\label{intro}
Lançado em novembro de 2004, o Portal da Transparência do Governo Federal é uma iniciativa da Controladoria-Geral da União que permite ao cidadão acompanhar a aplicação do dinheiro público federal. O site também oferece informações sobre diversos temas importantes para o controle social e tem como objetivo promover a transparência pública. \cite{guiaTransparencia2013}

Desde 27 de maio de 2010, para atender aos dispositivos previstos pela Lei Complementar nº 131/2009, o Portal da Transparência passou a disponibilizar dados sobre a execução orçamentária e financeira da Receita e da Despesa do Poder Executivo Federal com atualização diária. Os dados são fornecidos pela Secretaria do Tesouro Nacional (STN) e extraídos do Sistema Integrado de Administração Financeira do Governo Federal (Siafi). 

Criada para alterar a Lei de Responsabilidade Fiscal (Lei Complementar nº 101, de 4 de maio de 2000), no tocante à transparência da gestão, a Lei Complementar nº 131, de 27 de maio de 2009, entrou em vigor na data de sua publicação, em 28 de maio de 2009. A grande novidade trazida por ela foi a determinação de que a União, os Estados, o Distrito Federal e os Municípios disponibilizassem, em meio eletrônico e tempo real, informações pormenorizadas sobre sua execução orçamentária e financeira. Além disso, a LC nº 131/2009 tornou obrigatória a adoção, por todos os entes da Federação, de um sistema integrado de administração financeira e controle. Os sistemas adotados devem atender ao padrão mínimo de qualidade estabelecido pelo Poder Executivo da União no Decreto nº 7.185, de 27 de maio de 2010, e na Portaria MF nº 548, de 22 de novembro de 2010. \cite{guiaTransparencia2013} 

Segundo a legislação vigente (LC nº 131/2009 e Decreto nº 7.185/2010), devem ser disponibilizadas as seguintes informações relativas aos atos praticados pelas unidades gestoras, no decorrer da execução orçamentária e financeira:  

\begin{enumerate}[label=(\roman*)]
	\item quanto às despesas:
	\begin{enumerate}
		\item o valor do empenho, liquidação e pagamento;  
		
		\item o número do correspondente processo da execução, quando for o caso;  
		
		\item a classificação orçamentária, especificando a unidade orçamentária, função, subfunção, natureza da despesa e a fonte dos recursos que financiaram o gasto; d) a pessoa física ou jurídica beneficiária do pagamento, inclusive nos desembolsos de operações independentes da execução orçamentária, exceto no caso de folha de pagamento de pessoal e de benefícios previdenciários;  
		
		\item o procedimento licitatório realizado, bem como à sua dispensa ou inexigibilidade, quando for o caso, com o número do correspondente processo;  
		
		\item o bem fornecido ou serviço prestado, quando for o caso; 
	\end{enumerate}

	\item quanto à receita
	Deve-se publicar os valores de todas as receitas da unidade gestora, compreendendo no mínimo sua natureza, relativas a:  
	\begin{enumerate}
		\item previsão;  
		
		\item lançamento, quando for o caso; e  
		
		\item arrecadação, inclusive referente a recursos extraordinários. \cite{guiaTransparencia2013}  
	\end{enumerate}
\end{enumerate}

Diante de toda a legislação apresentada, como iniciativa de tornar pública informações relevantes da Administração Pública, percebemos claramente que esses ambientes disponibilizam um massivo volume de dados públicos estruturados, semiestruturados e não estruturados de interesse coletivo ou geral. Assim, torna-se um grande desafio a criação de aplicações capazes agregar e classificar esses dados, em uma velocidade apropriada a partir do enorme volume disponibilizado nos mais variados formatos.  

Segundo Vilarinho, 2017, hoje, a tecnologia nos permite coletar e armazenar cada vez maiores quantidades de dados e uma das formas de transformar esses dados em informações úteis é utilizando técnicas de Mineração de Dados (MD). A MD é uma técnica utilizada para a obtenção de informações a partir de grandes quantidades de dados. Ela é capaz de analisar diferentes tipos de elementos e encontrar diferentes tipos de relações entre eles. A aplicação da técnica de MD no portal da transparência a análise de diversos dados disponíveis na base de dados, que nos permitirão sistematizar uma série de informações pertinentes quanto ao não cumprimento das despesas públicas entre outras. O que temos observado nos últimos tempos, são a evolução dos recursos tecnológicos vivenciados pela sociedade atual e acontecem em proporção maior do que a capacidade que temos de assimilar essas mudanças. Um exemplo disso, é o surgimento e o crescimento da internet, que após deixar de ser utilizada somente por governos e a nível acadêmico, permitiu o amplo acesso a informações dos mais variados tipos, além de um vasto campo com recursos e serviços, como correio eletrônico, mensageiros instantâneos, compartilhamento de arquivo, redes sociais, entre outros, ou seja, nos traz um conceito de quebra de fronteiras da informação em larga escala e velocidade. Um termo muito próximo da nossa realidade são os “Agentes Inteligentes”, que, segundo os autores, trata-se de um recurso de programação que representa elementos autônomos, que têm a capacidade de manipular, trocar informações e também conhecimento, sendo assim, entidades que, através da codificação nelas inseridas, conseguem, com um grau de independência, executar as operações que lhes foram designadas. (Lima et al, 2014). 

\section{Motivação}
O que nos motiva a aprofundar o estudo dos dados disponíveis no Portal da Transparência são os benefícios que advêm da análise desses dados, bem como a fiscalização do cumprimento do orçamento público, com a mineração dos dados mais importantes, pode-se permitir o entendimento por parte da população a respeito da destinação dada aos recursos públicos, maior conhecimento dos diversos programas sociais oferecidos pelo Governo Federal e seus órgãos, entre outros. 



\section{Objetivos}
\subsection{Objetivo Geral }
Aplicar técnicas de mineração de dados e o uso de agentes inteligentes no Portal da Transparência, para classificar, minerar e prever os gastos diretos. 

\begin{comment}
Este é um comentário
\end{comment}

\subsection{Objetivos Específicos}
\begin{itemize}
\item Criar um agente inteligente para verificação da disponibilidade do Portal da Transparência; 

\item Buscar um determinado padrão de dados no portal de transparência referente aos gastos; 

\item Elaborar um agente inteligente que promova a sanitização dos dados no portal da transparência; 

\item Classificar os dados de relevância referente as despesas no portal da transparência. 

\end{itemize}
