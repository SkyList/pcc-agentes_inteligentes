% ---
% Capitulo de revisão de literatura
% ---


\chapter{Referencial Teórico}\label{referencial_teorico}

\section{Agentes inteligentes}

A globalização tem feito com que as organizações se coloquem em uma nova posição competitiva, onde o conhecimento e o comportamento dos seus colaboradores têm possibilitado vantagens competitivas. 

Para METAXIOTIS, 2004, a Inteligência Artificial tem recebido atenção durante as duas últimas décadas e tem sido amplamente aplicada em muitas áreas de negócio. As principais categorias analisadas são: Sistemas Especialistas (SE), Redes Neurais Artificiais (RNA) e Agentes Inteligentes (AI). 

\section{Sistemas Especialistas}

Considera-se que os sistemas especialistas surgiram, provavelmente, como uma área da Inteligência Artificial (IA) durante a década de 70, a partir do esforço de pesquisadores para desenvolver programas computacionais que pudessem raciocinar como humanos. Segundo WELBANK, 1983, um sistema especialista é um programa de computador que tem uma base de conhecimento sobre um domínio e utiliza o raciocínio para executar tarefas que especialistas humanos poderiam executar. 

\section{Portal da Transparência}

A temática transparência na gestão pública vem ganhando destaque nos últimos anos, sendo o acesso à informação reconhecido por importantes organismos da comunidade internacional como direito humano fundamental (CGU, 2013). 